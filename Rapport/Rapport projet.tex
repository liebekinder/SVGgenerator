
\documentclass[11pt]{report} % use larger type; default would be 10pt
%\documentclass[twocolumn]{book}
%\documentclass[twocolumn]{article}
%\documentclass[12pt]{book}
%\documentclass[11pt]{article}
%\documentclass[twocolumn, 12pt]{book}

%Pour écrire du français.
\usepackage[francais]{babel}
\usepackage[utf8]{inputenc}
\usepackage[T1]{fontenc}

%Police conseillée pour les pdf
\usepackage{lmodern}

%Choisir le format de sortie.
\usepackage{geometry} % to change the page dimensions
\geometry{a4paper} % or letterpaper (US) or a5paper or....
%\usepackage[top=2cm, bottom=2cm, left=2cm, right=2cm]{geometry}
% \geometry{margin=2in} % for example, change the margins to 2 inches all round
% \geometry{landscape} % set up the page for landscape
%   read geometry.pdf for detailed page layout information

\usepackage{setspace}

%Images incrustées
\usepackage{wrapfig}

%Tableaux
\usepackage{multirow}
\usepackage{slashbox}
\usepackage{colortbl}
\usepackage{color}


%euros
\usepackage{eurosym}

%Maths
\usepackage{amsmath}
\usepackage{amssymb}
\usepackage{mathrsfs}
\usepackage{amsthm}

%index
\usepackage{makeidx}

%url
\usepackage{url}

%Connaitre la dernière page
\usepackage{lastpage}



\usepackage{graphicx} % support the \includegraphics command and options

% \usepackage[parfill]{parskip} % Activate to begin paragraphs with an empty line rather than an indent

%%% PACKAGES
\usepackage{booktabs} % for much better looking tables
\usepackage{array} % for better arrays (eg matrices) in maths
\usepackage{paralist} % very flexible & customisable lists (eg. enumerate/itemize, etc.)
\usepackage{verbatim} % adds environment for commenting out blocks of text & for better verbatim
\usepackage{subfig} % make it possible to include more than one captioned figure/table in a single float

%%%Personal packages
\usepackage{moreverb}
\usepackage{latexsym}
\usepackage{framed}

\usepackage{xcolor}
\usepackage{listings}
\usepackage{caption}
\DeclareCaptionFont{white}{\color{white}}
\DeclareCaptionFormat{listing}{%
  \parbox{\textwidth}{\colorbox{gray}{\parbox{\textwidth}{#1#2#3}}\vskip-4pt}}
\captionsetup[lstlisting]{format=listing,labelfont=white,textfont=white}
\definecolor{mygreen}{rgb}{0,0.6,0}
\definecolor{mygray}{rgb}{0.5,0.5,0.5}
\definecolor{mymauve}{rgb}{0.58,0,0.82}
\lstset{frame=lrb,xleftmargin=\fboxsep,xrightmargin=-\fboxsep}

\lstset{ %
language=Java, % choose the language of the code
basicstyle=\footnotesize, % the size of the fonts that are used for the code
backgroundcolor=\color{white}, % choose the background color. You must add \usepackage{color}
showspaces=false, % show spaces adding particular underscores
showstringspaces=false, % underline spaces within strings
showtabs=false, % show tabs within strings adding particular underscores
frame=single, % adds a frame around the code
tabsize=4, % sets default tabsize to 2 spaces
captionpos=t, % sets the caption-position to bottom
breaklines=true, % sets automatic line breaking
breakatwhitespace=false, % sets if automatic breaks should only happen at whitespace
keywordstyle=\color{blue},
stringstyle=\color{mymauve},
rulecolor=\color{black},
extendedchars=true, belowcaptionskip=3ex,
escapeinside={\%*}{*)} % if you want to add a comment within your code
}



% These packages are all incorporated in the memoir class to one degree or another...

%%% HEADERS & FOOTERS
\usepackage{fancyhdr} % This should be set AFTER setting up the page geometry

\pagestyle{fancy} % options: empty , plain , fancy
\renewcommand{\headrulewidth}{0pt} % customise the layout...
\lhead{}\chead{}\rhead{}
\lfoot{}\cfoot{\thepage}\rfoot{}

%%% SECTION TITLE APPEARANCE
\usepackage{sectsty}
\allsectionsfont{\sffamily\mdseries\upshape} % (See the fntguide.pdf for font help)
% (This matches ConTeXt defaults)

%%% ToC (table of contents) APPEARANCE
\usepackage[nottoc,notlof,notlot]{tocbibind} % Put the bibliography in the ToC
\usepackage[titles,subfigure]{tocloft} % Alter the style of the Table of Contents
\renewcommand{\cftsecfont}{\rmfamily\mdseries\upshape}
\renewcommand{\cftsecpagefont}{\rmfamily\mdseries\upshape} % No bold!


%%% END Article customizations















%%% The "real" document content comes below...

\title{Rapport de projet}
\author{\textsc{Wollenburger} Antoine, \textsc{Chénais} Sébastien}
\date{} %Hide the date

\begin{document}
%Pied de page et entête (Un peu trop non ?) www.siteduzero.com/informatique/tutoriels/redigez-des-documents-de-qualite-avec-latex/les-styles-4
%Annule tous les trucs de base
\fancyhead[LE,CE,RE,LO,CO,RO]{}
\fancyfoot[LE,CE,RE,LO,CO,RO]{}

%Personnalisation
\fancyhead[LO, LE]{Rapport de programmation}
\fancyhead[RO,RE]{Page \thepage \ sur \pageref{LastPage}}
%\fancyfoot[LO,LE]{Université de \scshape{Nantes}}
%\fancyfoot[RO,RE]{Page \thepage \ sur \pageref{LastPage}}
%Lignes
\renewcommand{\headrulewidth}{0.4pt}
%\renewcommand{\footrulewidth}{0.4pt}

\maketitle

%Table des matières
\renewcommand{\contentsname}{Sommaire} % Dans le corps du document,avant la commande \tableofcontents.
\tableofcontents

\addcontentsline{toc}{chapter}{Introduction}
\chapter*{Introduction}
Il nous a été demandé d'écrire un compilateur, en OCAML, d'un langage de notre cru vers du svg (format pour décrire des images vectorielles). Pour mener à bien cette tâche, nous nous reposerons sur les notions que nous avons vu en cours.

\chapter{Évolution du projet}
\section{Définition d'une grammaire}
La première étape de ce projet a été la création d'un langage et sa grammaire associée. Même si le projet global se fera de manière incrémentale, il faut penser au fonctionnement global dès le début. 

Le fichier SVG est déclaré par l'instruction "drawing" suivi du nom du dessin et la taille du canevas sous la forme "[largeur , hauteur]". Les instructions sont incluses entre crochets.

Une instruction peut être de différents types:
\begin{description}
  \item[Une déclaration de variable]: on créé une variable en indiquant son type, son nom et enfin les valeurs qui servent à la construire.
  \item[Le dessin d'une variable]: pour les variables dont le type est adéquat, il s'agit de l'instruction qui déclenche l'affichage de ladite variable. La forme est "draw nomDeLaVariable".
  \item[Une conditionnelle]: il s'agit d'un test sur une ou plusieurs conditions qui influe sur la suite de l'exécution. Cette instruction est de la forme "if(-conditions-)\{\}else\{\}". Si la condition est vérifiée, alors le bloc suivant le if est exécuté, celui de else sinon.
  \item[Une boucle]: il s'agit d'une boucle sur condition. L'instruction est de la forme "while(-conditions-)\{\}". Le bloc entre crochets est répété jusqu'à ce que la condition soit invalidée.
  \item[Les fonctions et procédures]: il s'agit d'un appel a une fonction ou une procédure. L'instruction est de la forme "nomDeLaFonction (-paramètres-)". Pour les fonctions, il est possible de récupérer une valeur de retour.
\end{description}

\begin{lstlisting}[caption=Exemple de fichier source, language=C]
/* blabla bla commentaire
sur plusieurs lignes... */

drawing monDessin [256 , 256]
{
    //com sur une ligne
    /*function dessineSablier(Point centre, Number hauteur, Number largeur):void
    {
        Point HG(centre.x-hauteur/2,centre.y-largeur/2);
        Point HD(centre.x-hauteur/2,centre.y+largeur/2);
        Point BG(centre.x+hauteur/2,centre.y-largeur/2);
        Point BD(centre.x+hauteur/2,centre.y+largeur/2);

        Line diag2(BD,HG);
        Line haute(HD,HG);
        Line diag1(BG,HD);
        Line basse(BG,BD);

        draw diag2;
        draw diag1;
        draw haute;
        draw basse;
    }*/

	//Number test := 2;
    Point origine(50,50); //commentaire
    Point destination(100,100);
	
    Line ligne(origine,destination); //ligne
	/*if(test = 2){
		draw origine;
	}
	else{
		draw destination;
	}
	while(test > 0){
		draw destination;
		test := test -1;
	}*/
    //origineSablier.x := 40;
    //dessineSablier(c,20,20);
    draw ligne;
}
\end{lstlisting}
%TODO grammaire

\section{Lexer}
Nous avons alors écrit l'analyseur syntaxique, comprenant les mots de notre grammaire. Cette étape est de loin la plus facile.

\section{Parser}
Ensuite, nous avons transformé le fichier d'entrée en un arbre binaire, modèle de représentation intermédiaire, qui nous permettra l'écriture du fichier svg. %TODO Dessin d'arbre en exemple.
Nous avons décidé d'implémenter nos arbres comme suit : 

\begin{lstlisting}[caption=Type arbre, language=caml]
type operation =
    Drawing
  | DrawingSize
  | BlocEmbrace
  | Declaration
  | BlocBrace
  | Parameters
  | Point
  | Line
  | Instruction
  | Draw
  | Var of (string)
  | Number of (int);;
  
type t_arbreB = Empty | Node of node
        and node = { value: operation; left: t_arbreB; right: t_arbreB };;
\end{lstlisting}

\section{Et le bonheur arrive.}
Viens alors la partie de loin la plus fastidieuse. En effet, pour transformer cet arbre binaire en svg, nous avons besoin d'une implémentation d'un automate à pile. Pour ceci point de choix. Nous passerons par une table shift-reduce. A la vue de la taille du vocabulaire, cette table est tout sauf petite.
%TODO table s-r

\end{document}